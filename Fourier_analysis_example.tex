\documentclass{article}
\usepackage{graphicx}%needs this for includesgraphics command
\usepackage{amsmath}
\usepackage{extarrows}
\begin{document}
\title{\LaTeX \ Practice paper, Fourier Analysis from Wikipedia}
\author{Brandon Goodyear}
\maketitle
\tableofcontents

\section{Definitions}
Fourier analysis is the study of the way general functions may be represented or approximated by sums of simpler trigonometic functions. \\
It is named after Joseph Fourier, who showed that representing a function as a sum of trigonometic funtions greatly simplifies the sudy of heat transfer. \\
The process of decomposing a function into oscillatory components is often claled Fourier analysis, while the operation of rebuilding the fuction from these pieces is known as \textbf{Fourier synthesis}.
The decomposition process itself is called a \textit{Fourier transformation}.

\section{Variants of Fourier analysis}
\subsection{general}
\textbf{Fourier transform} refers to the transform of functions of a continous real argument. They produce a continous function of frequency, known as \textit{frequency distribution}. If a fucntion is tranformed, that transformed funciton can then be transformed back to the original. \\
When the domain of the inital fucniton is $(t)$, and the domain of the output function is ordanary frequency, the transform of function $s(t)$ at frequency $f$ is given by the complex number:
\begin{equation*}
S(f)=\int_{-\infty}^{\infty}{s(t) \cdot e^{-i2\pi ft}}dt
\end{equation*}
Evaluating this quanity for all values of $f$ produces the \textit{frequency-domain} function. Then $s(t)$ can be represented as a recombination of complex exponentials of all possible frequencies:
\begin{equation*}
s(t)=\int_{-\infty}^{\infty}{S(f) \cdot e^{i2\pi ft}}df \text{,}
\end{equation*}
which is the inverse transform formula. The complex number, $S(f)$, convey both amplitude and phase of frequency $f$.

\subsection{Fourier Series}
The Fourier transform of a periodic function, $s_P(t)$, with period $P$, becomes a Diac comb function, modulated by a sequence of complex coefficeints:

\begin{equation*}
S[k]=\frac{1}{P}\int_{p}^{}{s_P(t) \cdot e^{-i2\pi \frac{k}{P}t}}dt
\end{equation*}

for all integer values of $k$, and where $\int_{P}$ is the integral over any interval of length $P$.\\
The inverse transform, known as \textbf{Fourier series}, is a representation of $s_p(t)$ in terms of a summation of a potentially infinite number of harmonically related sinusoids or complex exponential functions, each with an amplitude and phase specified by one of the coeffients:

\begin{equation*}
s_p(t)=\sum_{k=-\infty}^{\infty}{S[k] \cdot e^{i2\pi \frac{k}{P}t}} 
\xLongleftrightarrow[{}]{\text{$F$}}
\sum_{k=-\infty}^{+\infty}{S[k]\delta \left(f-\frac{k}{P}\right)}
\end{equation*}

When $S_p(t)$,is expressed as a periodic summation of another function,s(t):

\begin{equation*}
s_P(t)\overset{\Delta}{=}\sum_{m=-\infty}^{\infty}s(t-mP)\text{,}
\end{equation*}

the coefficents are proportianal to samples of $S(f)$ at discrete intervals of $\frac{1}{P}$

\begin{equation*}
S[k]=\frac{1}{P}\cdot S\left(\frac{k}{P}\right)
\end{equation*}

A sufficient condition for recovering $s(t)$ (and therefore $S(f)$) from just these samples is that the non-zero portion of $s(t)$ be confined to a known interval of duration $P$, which is the frequency domain dual of Nyquist-Shannon sampling theroem. 

\subsection{Discrete-time Fourier transform (DTFT)}

\end{document}



