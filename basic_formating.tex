\documentclass{article}
\begin{document}
\title{Formating \LaTeX \  in TeXworks}
\author{Brandon Goodyear}
\date{\today}
\maketitle
\tableofcontents
\begin{flushleft}This will produce \textit{italicized} text.\end{flushleft} %why is this indented w/o flushleft?why extra line w/flushleft?
This will produce \textbf{blod-faced} text.\\
This will produce \textsc{small caps} text.\\
This will produce \texttt{typerwriter} font.\\
Please excuse my dear aunt Sally. \\
Please excuse my \begin{large} dear aunt Sally \end{large}.\\
Please excuse my \begin{Large}dear aunt Sally \end{Large}.\\
Please excuse my \begin{huge}dear aunt Sally \end{huge}.\\
Please excuse my \begin{Huge}dear aunt Sally \end{Huge}.\\
Please excuse my \begin{small}dear aunt Sally \end{small}.\\
Please excuse my \begin{tiny}dear aunt Sally \end{tiny}.\\
\begin{center} This is centered. \end{center}
\begin{flushleft} This is left-justified \end{flushleft}
\begin{flushright} This is rigt-justified \end{flushright}

\section{Linear functions}
	\subsection{Slope-intercept form}
	The slope-intercept form of a linear function is givne by $y=ax+b$
	\subsection{Standar form}
	\subsection{Point-Slope form}
\section{Quadratic functions}
	\subsection{Vertex form}
	\subsection{Standard form}
	\subsection{Factor form}




\end{document}